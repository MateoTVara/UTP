\documentclass[12pt]{article}

%===============================
%
%          📦 Paquetes
%
%===============================

\usepackage[a4paper, top=2cm, bottom=2cm, left=2.5cm, right=2.5cm]{geometry}
\usepackage[spanish]{babel}
\usepackage[utf8]{inputenc}
\usepackage{amsmath}
\usepackage{multicol}
\usepackage{graphicx}
\usepackage{hyperref}
\usepackage{booktabs}
\usepackage{pgfplots}
\pgfplotsset{compat=1.18}

\title{
  \vspace{2cm}
  \pagenumbering{gobble}
  \includegraphics[width=5cm]{../assets/logo-utp.png} \\
  \vspace{1cm}
  \textbf{Universidad Tecnológica del Perú} \\
  \vspace{2cm}
  \textbf{Innovación y Transformación Digital} \\
  \vspace{1cm}
  \large \textbf{S04 - Tarea: Casos de Éxito en Latinoamérica}
}
\author{
  \begin{tabular}{ll}
    \textbf{Abad Reyes, Manuel Emilio} & \texttt{U21221095} \\
    \textbf{Rojas Rioja, Jasmin} & \texttt{U22327848} \\
    \textbf{Mercedes Bolo, Deivi Steve} & \texttt{U22230489} \\
    \textbf{Oribe Vigo, Aimar Glover} & \texttt{U24232165} \\
    \textbf{Torres Vara, Mateo Nicolas} & \texttt{U24308542} \\
  \end{tabular} \\\\
  \texttt{Sección 37593}
}



\begin{document}
\maketitle
\begin{center}

  Docente: Brangel Omar Saucedo Vasquez

\end{center}

%======================================
%
%          📚 Inicio del documento
%
%======================================

\newpage
\pagenumbering{arabic}

\section*{Introducción}
\noindent En el presente documento se aplica una metodología de innovación al problema identificado en el Avance de Proyecto Final 1: la brecha digital que enfrenta el BBVA en Perú. Este problema afecta principalmente a adultos mayores y a personas en zonas rurales que tienen un acceso limitado a internet y a las aplicaciones bancarias, lo que restringe su inclusión financiera y mantiene su dependencia de las oficinas físicas.

\section*{Metodología seleccionada}
\noindent Se eligió la metodología \textbf{Design Thinking}, dado que se centra en comprender las necesidades de los usuarios, empatizar con sus realidades y diseñar soluciones inclusivas y viables.

\subsection*{Empatizar}
\noindent Se analizó a los adultos mayores, quienes suelen sentir desconfianza y miedo a cometer errores o ser víctimas de fraudes, y a los clientes en zonas rurales, quienes enfrentan dificultades de conectividad y baja experiencia en el uso de aplicaciones móviles.

\subsection*{Definir}
\noindent El problema se enuncia de la siguiente manera:  
\emph{“Los clientes adultos mayores y de zonas rurales tienen barreras tecnológicas y de confianza que les impiden usar de manera sencilla y segura los servicios financieros digitales de BBVA, lo que limita su inclusión financiera y aumenta su dependencia de las oficinas físicas.”}

\subsection*{Idear}
\noindent Se propusieron alternativas como:
\begin{itemize}
  \item Una aplicación con modo simplificado para operaciones frecuentes.
  \item Un asistente virtual con Procesamiento de Lenguaje Natural (NLP) en español.
  \item Funcionalidades offline parciales.
  \item Opciones de accesibilidad (alto contraste, texto grande, lectores de pantalla).
  \item Programas de educación financiera digital básica.
\end{itemize}

\subsection*{Prototipar}
\noindent Se diseñó el prototipo conceptual \textbf{BBVA Inclusión Digital}, con una interfaz simple que incluye tres botones principales: \emph{consultar saldo}, \emph{transferir dinero} y \emph{pagar servicios}. Cada botón se acompaña de íconos intuitivos.  
Además, incorpora un asistente por voz con NLP y un modo de accesibilidad para personas con discapacidad visual.

\subsection*{Testear}
\noindent Se propone un piloto ficticio con adultos mayores de Lima y usuarios de zonas rurales, midiendo:
\begin{itemize}
  \item Porcentaje de operaciones completadas sin ayuda.
  \item Tiempo promedio para realizar tareas básicas.
  \item Nivel de confianza y satisfacción después del uso.
\end{itemize}

\section*{Propuesta de solución final}
\noindent La solución propuesta es \textbf{BBVA Inclusión Digital}, una aplicación ligera y accesible, basada en Inteligencia Artificial y Cloud Computing. Esta propuesta permitiría personalizar la experiencia de los usuarios, reducir las barreras tecnológicas y promover la inclusión financiera, reforzando la imagen de BBVA como un banco innovador y socialmente responsable.

\section*{Conclusiones}
\noindent La aplicación del Design Thinking al caso del BBVA muestra cómo una metodología centrada en el usuario permite desarrollar soluciones inclusivas para reducir la brecha digital. La propuesta planteada atiende tanto las necesidades tecnológicas como las de confianza de los clientes, lo que contribuiría a una mayor inclusión financiera en el país.

\section*{Recursos y Créditos}
\begin{itemize}
  \item INEI, “Digitalización e inclusión financiera en Perú,” Revista Moneda N.º 197, Banco Central de Reserva del Perú, marzo 2024. Disponible en: \url{https://www.bcrp.gob.pe/docs/Publicaciones/Revista-Moneda/moneda-197/moneda-197-02.pdf}
  \item \textbf{Código fuente:} \href{https://github.com/MateoTVara/C08-InnovacionYTransformacionDigital}{Repositorio GitHub - Innovacion y Transformacion Digital}
  \item \textbf{Carátula por:} \href{https://github.com/1nfinit0}{1nfinit0 en GitHub}
\end{itemize}




\end{document}