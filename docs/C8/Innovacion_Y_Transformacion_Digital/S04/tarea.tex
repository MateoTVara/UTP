\documentclass[12pt]{article}

%===============================
%
%          📦 Paquetes
%
%===============================

\usepackage[a4paper, top=2cm, bottom=2cm, left=2.5cm, right=2.5cm]{geometry}
\usepackage[spanish]{babel}
\usepackage[utf8]{inputenc}
\usepackage{amsmath}
\usepackage{multicol}
\usepackage{graphicx}
\usepackage{hyperref}
\usepackage{booktabs}
\usepackage{pgfplots}
\pgfplotsset{compat=1.18}

\title{
  \vspace{2cm}
  \pagenumbering{gobble}
  \includegraphics[width=5cm]{./assets/logo-utp.png} \\
  \vspace{1cm}
  \textbf{Universidad Tecnológica del Perú} \\
  \vspace{2cm}
  \textbf{Innovación y Transformación Digital} \\
  \vspace{1cm}
  \large \textbf{S04 - Tarea: Casos de Éxito en Latinoamérica}
}
\author{
  \begin{tabular}{ll}
    \textbf{Torres Vara, Mateo Nicolas} & \texttt{U24308542} \\
    \textbf{Abad Reyes, Manuel Emilio} & \texttt{U21221095} \\
    \textbf{Rojas Rioja, Jasmin} & \texttt{U22327848} \\
    \textbf{Oribe Vigo, Aimar Glover} & \texttt{-} \\
  \end{tabular} \\\\
  \texttt{Sección 37593}
}



\begin{document}
\maketitle
\begin{center}

  Docente: Brangel Omar Saucedo Vasquez

\end{center}

%======================================
%
%          📚 Inicio del documento
%
%======================================

\newpage
\section*{Blockchain - Bolsa de Comercio de Santiago (Chile)}
\noindent La Bolsa implementó una plataforma basada en blockchain (Hyperledger Fabric), con el apoyo de IBM y la Fundación Linux, para automatizar operaciones de ventas cortas y préstamo de valores mediante contratos inteligentes. Gracias a esta innovación, los tiempos de procesamiento pasaron de 4 días a solo 4 minutos, generando mayor eficiencia, confiabilidad y reducción de costos operativos. Además, el uso de blockchain fortaleció la transparencia en las transacciones, impulsando la confianza de los inversionistas y posicionando a Chile como pionero en la región en el uso de esta tecnología para el mercado financiero.

\section*{Cloud Computing - SCM Latam}
\noindent SCM Latam utiliza servicios de Google Cloud (Compute Engine, Cloud Run, Cloud Functions) para optimizar la gestión del personal y asegurar la disponibilidad de la infraestructura en un 99.9 \%. Esto permite acelerar el procesamiento de grandes volúmenes de datos y aumentar la productividad de sus clientes en toda Latinoamérica. Gracias a la escalabilidad de la nube, la empresa logró reducir tiempos de implementación, mejorar la seguridad en sus operaciones y ofrecer soluciones más flexibles a empresas de distintos tamaños.

\section*{Inteligencia Artificial (IA)}
\noindent En un listado de 90 casos de uso en Latinoamérica, destacan aplicaciones que aprovechan la IA generativa para optimizar procesos financieros y de servicios:
\begin{itemize}
    \item \textbf{ALBO (México)}: Plataforma financiera digital que utiliza IA para ofrecer servicios bancarios personalizados, mejorar la experiencia del usuario y optimizar la gestión de riesgos, facilitando la inclusión financiera.
    \item \textbf{BANCO COVALTO (México)}: Redujo en más del 90 \% los tiempos de respuesta para otorgar créditos gracias a la IA generativa, lo que favorece a las pymes con acceso más rápido a financiamiento.
    \item \textbf{BEWE (Colombia)}: Asistente basado en IA generativa que mejora la fidelización y conversión de clientes para pymes del sector belleza y bienestar, apoyando la digitalización de negocios locales.
\end{itemize}

\newpage
\section*{IoT (Internet de las Cosas) – Smart City y recolección inteligente (Argentina)}
\noindent La ciudad de San Nicolás colaboró con Telefónica IoT para transformarse en una ciudad inteligente mediante la implementación de sensores y plataformas de monitoreo. Entre los resultados más destacados:
\begin{itemize}
    \item Patrullaje urbano pasó de cubrir el 80\% al 100\% de las calles, aumentando la seguridad ciudadana.
    \item Reducción del tiempo de respuesta de patrullas en un 15\%, mejorando la eficiencia en emergencias.
    \item Mayor control en las rutas de camiones recolectores y optimización de la recolección de residuos, logrando un uso más eficiente de los recursos y una recuperación rápida de la inversión realizada.
\end{itemize}

\section*{Ciberseguridad - Truora (Colombia)}
\noindent Fundada en 2019, Truora ha mejorado significativamente la seguridad digital para empresas de la región mediante la verificación de identidad y el uso de reconocimiento facial. Entre sus principales logros destacan:
\begin{itemize}
    \item Rappi: reducción del 99\% en perfiles fraudulentos de rappitenderos, fortaleciendo la confianza en la plataforma.
    \item Addi: disminución del riesgo de fraude en un 80\%, validando más de 300 000 documentos con procesos rápidos y seguros.
\end{itemize}
La compañía, reconocida en LinkedIn Top Startups, recibió US \$15 millones para expandir su alcance en Latinoamérica y seguir impulsando una cultura de seguridad digital en la región.

\newpage
\section*{Recursos y créditos}

\begin{itemize}
    \item \textbf{Código fuente:} \href{https://github.com/MateoTVara/C08-InnovacionYTransformacionDigital}{Repositorio GitHub - Innovacion y Transformacion Digital}
    \item \textbf{Carátula por:} \href{https://github.com/1nfinit0}{1nfinit0 en GitHub}
    \item \textbf{Blockchain:} \href{https://news.america-digital.com/caso-de-exito-primera-aplicacion-blockchain-en-mercado-de-valores-de-latam-conferencia-cio-bolsa-de-santiago/?utm_source=chatgpt.com}{Caso de éxito - Bolsa de Santiago}
    \item \textbf{Cloud Computing:} \href{https://cloud.google.com/customers/scm-latam?hl=es-419&utm_source=chatgpt.com}{SCM Latam en Google Cloud}
    \item \textbf{Inteligencia Artificial:} \href{https://blog.google/intl/es-419/actualizaciones-de-producto/en-la-nube/google-cloud-90-casos-de-ia-en-latinoamerica-que-inspiran-innovacion/?utm_source=chatgpt.com}{90 casos de IA en Latinoamérica - Google Cloud}
    \item \textbf{IoT:} \href{https://sannicolasciudad.gob.ar/areas/innovacion-y-transformacion-digital/ciudadinteligente}{Smart City San Nicolás}
    \item \textbf{Ciberseguridad:} \href{https://elpais.com/america-colombia/branded/los-lideres-de-colombia/2024-12-05/daniel-bilbao-el-caleno-que-esta-transformando-la-cultura-de-seguridad-digital-en-america-latina.html?utm_source=chatgpt.com}{Truora - Seguridad digital en Colombia}
\end{itemize}


\end{document}