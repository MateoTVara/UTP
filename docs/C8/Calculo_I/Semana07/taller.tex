\documentclass[12pt]{article}

%===============================
%
%          📦 Paquetes
%
%===============================

\usepackage[a4paper, top=2cm, bottom=2cm, left=2.5cm, right=2.5cm]{geometry}
\usepackage[spanish]{babel}
\usepackage[utf8]{inputenc}
\usepackage{amsmath}
\usepackage{multicol}
\usepackage{graphicx}
\usepackage{hyperref}
\usepackage{booktabs}
\usepackage{pgfplots}
\pgfplotsset{compat=1.18}
\usepackage{tikz}

\title{
  \vspace{2cm}
  \pagenumbering{gobble}
  \includegraphics[width=5cm]{../assets/logo-utp.png} \\
  \vspace{1cm}
  \textbf{Universidad Tecnológica del Perú} \\
  \vspace{2cm}
  \textbf{Cálculo I} \\
  \vspace{1cm}
  \large \textbf{Taller 2}
}
\author{
  \textbf{Torres Vara, Mateo Nicolas} - \texttt{U24308542} \\
  \texttt{Sección 32384}
}



\begin{document}
\maketitle
\begin{center}

  Docente: Victor Johnny Papuico Bernardo

\end{center}

%======================================
%
%          📚 Inicio del documento
%
%======================================

\newpage
\pagenumbering{arabic}

\section*{Ejercicio 1}
\noindent Determine el valor de los siguientes límites:
\begin{center}
\[
\begin{array}{c|c|c}
  \begin{array}{c}
    \lim_{x \to \infty} \dfrac{6x^2 + 5x^3 - 4}{2x^3 - 3 - x^2} \\\\
    \lim_{x \to \infty} \dfrac{\dfrac{6x^2}{x^3} + \dfrac{5x^3}{x^3} - \dfrac{4}{x^3}}
                              {\dfrac{2x^3}{x^3} - \dfrac{3}{x^3} - \dfrac{x^2}{x^3}} \\\\
    \lim_{x \to \infty} \dfrac{\dfrac{6}{x} + 5 - \dfrac{4}{x^3}}
                              {2 - \dfrac{3}{x^3} - \dfrac{1}{x}} \\\\
    \dfrac{0 + 5 - 0}{2 - 0 - 0} = \dfrac{5}{2}
  \end{array}
  &
  \begin{array}{c}
    \lim_{x \to \infty} \dfrac{\sqrt{9x^4-1}+5x^2}{\sqrt{x^4-2}+x} \\\\
    \lim_{x \to \infty} \dfrac{\sqrt{\dfrac{9x^4}{x^4} - \dfrac{1}{x^4}} + \dfrac{5x^2}{x^2}}
                              {\sqrt{\dfrac{x^4}{x^4} - \dfrac{2}{x^4}} + \dfrac{x}{x^2}} \\\\
    \lim_{x \to \infty} \dfrac{\sqrt{9 - \dfrac{1}{x^4}} + 5}
                              {\sqrt{1 - \dfrac{2}{x^4}} + \dfrac{1}{x}} \\\\
    \dfrac{\sqrt{9 - 0} + 5}{\sqrt{1 - 0} + 0} = \dfrac{3 + 5}{1 + 0} = 8
  \end{array}
  &
  \begin{array}{c}
    \lim_{x \to \infty} \dfrac{\sqrt{81x^6 + 8} - 5x^3}{\sqrt{4x^6 -x} + 1} \\\\
    \lim_{x \to \infty} \dfrac{\sqrt{\dfrac{81x^6}{x^6} + \dfrac{8}{x^6}} - \dfrac{5x^3}{x^3}}
                              {\sqrt{\dfrac{4x^6}{x^6} - \dfrac{x}{x^6}} + \dfrac{1}{x^3}} \\\\
    \lim_{x \to \infty} \dfrac{\sqrt{81 + 0} - 5}
                              {\sqrt{4 - 0} + 0} \\\\
    \dfrac{9 - 5}{2 + 0} = \dfrac{4}{2} = 2
  \end{array}
\end{array}
\]
\end{center}

\section*{Ejercicio 2}
\noindent Determine el valor de los siguientes límites:
\[
\begin{array}{l|c}
  \begin{array}{l}
    \lim_{x \to 4} \dfrac{\sqrt{2x+1} - 3}{2x^2-x-28} = \dfrac{0}{0} \\\\
    \lim_{x \to 4} \dfrac{\sqrt{2x+1}-3}{2x^2-x-28} \cdot \dfrac{\sqrt{2x+1}+3}{\sqrt{2x+1}+3} \\\\
    \lim_{x \to 4} \dfrac{2x-8}{(x-4)(2x+7)(\sqrt{2x+1}+3)} \\\\
    \lim_{x \to 4} \dfrac{2}{(2(4)+7)(\sqrt{2(4)+1}+3)} = \dfrac{1}{45} \\\\
  \end{array}
  &
  \begin{array}{c|c c|c}
    & 2 & -1 & -28 \\
    \hline
    4 &  & 8  & 28 \\
    \hline
    & 2 & 7  & 0 \\
  \end{array}
\end{array}
\]

\[
\begin{array}{l|c}
  \begin{array}{l}
    \lim_{x \to -1} \dfrac{2x^3 - 7x^2 - 5x + 4}{x^2 + 6x + 5} = \dfrac{0}{0} \\\\
    \lim_{x \to -1} \dfrac{(x+1)(2x^2 - 9x + 4)}{(x+1)(x+5)} \\\\
    \lim_{x \to -1} \dfrac{2x^2 - 9x + 4}{x+5} \\\\
    \lim_{x \to -1} \dfrac{2(-1)^2 - 9(-1) + 4}{-1 + 5} = \dfrac{15}{4} \\\\
  \end{array}
  &
  \begin{array}{c}
    \begin{array}{c|c c c|c}
      & 2 & -7 & -5 & 4 \\
      \hline
      -1 &  & -2 & 9 & -4 \\
      \hline
      & 2 & -9 & 4 & 0 \\
    \end{array} \\\\
    \begin{array}{c|c c|c}
      & 1 & 6 & 5 \\
      \hline
      -1 &  & -1 & -5 \\
      \hline
      & 1 & 5 & 0 \\
    \end{array}
  \end{array}
\end{array}
\]

\[
\begin{array}{l|c}
  \begin{array}{l}
    \lim_{x \to 1} \dfrac{3x^2 - x - 2}{\sqrt{7-3x} - 2} = \dfrac{0}{0} \\\\
    \lim_{x \to 1} \dfrac{3x^2 - x - 2}{\sqrt{7-3x} - 2} \cdot \dfrac{\sqrt{7-3x} + 2}{\sqrt{7-3x} + 2} \\\\
    \lim_{x \to 1} \dfrac{(3x + 2)(\sqrt{7-3x} + 2)}{-3} \\\\
    \lim_{x \to 1} \dfrac{(3(1) + 2)(\sqrt{7-3(1)} + 2)}{-3} = -\dfrac{20}{-3} \\\\
  \end{array}
  &
  \begin{array}{c|c c|c}
    & 3 & -1 & -2 \\
    \hline
    1 &  & 3 & 2 \\
    \hline
    & 3 & 2 & 0 \\
  \end{array}
\end{array}
\]






\section*{Ejercicio 3}
\noindent Dada la siguiente función:

\[
f(x) = 
\left\{
  \begin{array}{cll}
    \dfrac{6x^3 + 23x^2 + 26x - 8}{x^2 - x -2} & \text{ , si } & x < 2 \\\\
    2 & \text{ , si } & x = 2 \\\\
    \dfrac{x-2}{\sqrt{x-1} - 1} & \text{ , si } & x > 2
  \end{array}
\right.
\]

\noindent ¿Es continua en $x=2$ ?

\[
\begin{array}{l|c}
  \begin{array}{l}
    \lim_{x \to 2} \dfrac{6x^2 + 11x + 4}{x + 1} \\\\
    \dfrac{6(2)^2 - 11(2) + 4}{3} \\\\
    \dfrac{24 - 22 + 4}{3} = \dfrac{6}{3} = 2
  \end{array}
  &
  \begin{array}{c|c c c|c}
    & 6 & -23 & 26 & -8 \\
    \hline
    2 &  & 12 & -22 & 8 \\
    \hline
    & 6 & -11 & 4 & 0 \\
  \end{array}
\end{array}
\]

\vspace{1cm}

\[
\begin{array}{l|c}
  \begin{array}{l}
    \lim_{x \to 2} \dfrac{(x-2) (\sqrt{x - 1} + 1)}{x-2}\\\\
    (\sqrt{2 - 1} + 1) = 2
  \end{array}
  &
  \begin{array}{c|c c|c}
    & 1 & -1 & -2 \\
    \hline
    2 &  & 2 & 2 \\
    \hline
    & 1 & 1 & 0 \\
  \end{array}
\end{array}
\]

\subsection*{Respuesta:} \noindent Si es continua en $x=2$





\section*{Ejercicio 4}
\noindent Dada la siguiente función:
\[
f(x) = 
\left\{
  \begin{array}{c l l}
    \dfrac{2x^3 + 7x^2 - 14x + 5}{x-1} & \text{ , si } & x < 1 \\\\
    6 & \text{ , si } & x = 1 \\\\
    \dfrac{\sqrt{17 - x} + 8}{x^2 + 1} & \text{ , si } & x > 1
  \end{array}
\right.
\]

\noindent ¿Es continua en $x=1$ ?

\[
\begin{array}{l|c}
  \begin{array}{l}
    \lim_{x \to 1} 2x^2 + 9x - 5 \\\\
    2 + 9 - 5 = 6 
  \end{array}
  &
  \begin{array}{c|c c c|c}
    & 2 & 7 & -14 & -5 \\
    \hline
    1 &  & 2 & 9 & -5 \\
    \hline
    & 2 & 9 & -5 & 0 \\
  \end{array}
\end{array}
\]

\subsection*{Respuesta:} \noindent Si es continua en $x=1$


\end{document}