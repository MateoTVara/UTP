\documentclass[12pt]{article}

%===============================
%
%          📦 Paquetes
%
%===============================

\usepackage[a4paper, top=2cm, bottom=2cm, left=2.5cm, right=2.5cm]{geometry}
\usepackage[spanish]{babel}
\usepackage[utf8]{inputenc}
\usepackage{amsmath}
\usepackage{multicol}
\usepackage{graphicx}
\usepackage{xcolor} % added for \fcolorbox
\usepackage{hyperref}
\usepackage{booktabs}
\usepackage{pgfplots}
\pgfplotsset{compat=1.18}

\title{
  \vspace{2cm}
  \pagenumbering{gobble}
  \includegraphics[width=5cm]{../assets/logo-utp.png} \\
  \vspace{1cm}
  \textbf{Universidad Tecnológica del Perú} \\
  \vspace{2cm}
  \textbf{Investigación Operativa} \\
  \vspace{1cm}
  \large \textbf{S18 - Evaluación Final}
}
\author{
  \textbf{Torres Vara, Mateo Nicolas} - \texttt{U24308542} \\
  \texttt{Sección 36373}
}



\begin{document}
\maketitle
\begin{center}

  Docente: Alberto Andre Reyna Alcantara

\end{center}

%======================================
%
%          📚 Inicio del documento
%
%======================================

\newpage
\section*{Caso 1 - Programación multiobjetivo y entera}

\noindent Una fábrica produce dos tipos de sillas: Ejecutiva y Económica. Cada silla ejecutiva deja una utilidad de 
S/ 120, y cada silla económica de S/ 80. Para producir una silla ejecutiva se necesitan 3 horas de mano de obra y 
2m² de madera, mientras que una económica requiere 2 horas y 1m². \\
\noindent Se dispone de 120 horas de trabajo y 90m² de madera. La empresa tiene las siguientes metas en orden de prioridad:

\begin{itemize}
  \item[] \textbf{Meta 1}: Alcanzar una utilidad mínima de S/ 4000.
  \item[] \textbf{Meta 2}: Usar al menos el 95\% de las horas disponibles.
  \item[] \textbf{Meta 3}: No usar más del 90\% de la madera disponible. 
\end{itemize}

\noindent Además la producción debe ser en unidades enteras.

\subsection*{Modelo en Lingo}

\begin{center}
  \setlength{\fboxsep}{4pt}  % padding
  \setlength{\fboxrule}{1pt} % border thickness

  \fbox{\includegraphics[width=0.8\textwidth]{assets/case1-model.PNG}}
\end{center}

\noindent Se puede notar que la solución cambia al forzar enteros:

\begin{center}
  \setlength{\fboxsep}{4pt}  % padding
  \setlength{\fboxrule}{1pt} % border thickness

  \begin{minipage}[t]{0.45\textwidth}
    \centering
    \fbox{\includegraphics[width=\linewidth]{assets/case1-results-noentero.PNG}}
    \\[4pt]\small Modelo Sin Enteros
  \end{minipage}\hfill
  \begin{minipage}[t]{0.45\textwidth}
    \centering
    \fbox{\includegraphics[width=\linewidth]{assets/case1-results-entero.PNG}}
    \\[4pt]\small Modelo Con Enteros
  \end{minipage}
\end{center}

\newpage
\subsection*{Comprobamos las metas}

\noindent \textbf{Variables de decisión:} \\
\begin{center}
  \noindent A = Sillas Ejecutivas producidas = 26 \\
  \noindent B = Sillas Económicas producidas = 19 \\
\end{center}

\[
  120A + 80B = 120(26) + 80(19) = 3120 + 1520 = 4640 \quad \text{(Meta 1 cumplida)}
\]
\[
  3A + 2B = 3(26) + 2(19) = 78 + 38 = 116 \quad \text{(Meta 2 cumplida, 116/120 = 96.67\%)}
\]
\[
  2A + B = 2(26) + 19 = 52 + 19 = 71 \quad \text{(Meta 3 cumplida, 71/90 = 78.89\%)}
\]

\subsection*{Conclusiones}
\begin{itemize}
  \item La producción óptima es de 26 sillas ejecutivas y 19 sillas económicas.
  \item Se cumplen todas las metas establecidas por la empresa.
  \item La producción en unidades enteras afecta la solución óptima, pero aún así se logran cumplir las metas.
\end{itemize}





\newpage
\subsection*{Caso 2 - Programación Binaria y dinámica}
\noindent Una empresa debe elegir entre 5 proyectos posibles. Cada proyecto tiene un costo y un retorno estimado, 
pero algunos proyectos son excluyentes o dependientes entre sí. Se dispone de un presupuesto máximo de S/ 10,000.

\begin{center}
\begin{tabular}{lccc}
\toprule
\textbf{Proyecto} & \textbf{Costo (S/)} & \textbf{Retorno (S/)} \\
\midrule
P1 & 3,000 & 5,000 \\
P2 & 4,000 & 7,000 \\
P3 & 2,000 & 3,000 \\
P4 & 5,000 & 9,000 \\
P5 & 1,000 & 1,500 \\
\bottomrule
\end{tabular}
\end{center}

\noindent\textbf{Reglas:}
\begin{itemize}
  \item[•] Si se elige P1, no se puede elegir P3.
  \item[•] P2 y P4 no pueden ser elegidos simultáneamente.
  \item[•] P5 solo puede ser elegido si se elige P2.
\end{itemize}

\noindent Se quiere maximizar el retorno total sin exceder el presupuesto.



\subsection*{Modelo en Lingo y Resultados}
\begin{center}
  \setlength{\fboxsep}{4pt}  % padding
  \setlength{\fboxrule}{1pt} % border thickness

  \begin{minipage}[t]{0.45\textwidth}
    \centering
    \fbox{\includegraphics[width=\linewidth]{assets/case2-binarymodel.PNG}}
    \\[4pt]\small Modelo en Lingo
  \end{minipage}\hfill
  \begin{minipage}[t]{0.45\textwidth}
    \centering
    \fbox{\includegraphics[width=\linewidth]{assets/case2-binaryresults.PNG}}
    \\[4pt]\small Resultados
  \end{minipage}
\end{center}



\subsection*{Comprobación mediante programación dinámica}

\begin{center}
  \setlength{\fboxsep}{4pt}  % padding
  \setlength{\fboxrule}{1pt} % border thickness

  \fbox{\includegraphics[width=0.8\textwidth]{assets/case2-dinamic.png}}
\end{center}



\subsection*{Conclusiones}
\begin{itemize}
  \item La selección óptima de proyectos es P1 y P4, con un retorno total de S/ 14000.
  \item Se cumplen todas las restricciones de exclusión y dependencia entre proyectos.
  \item La programación binaria y dinámica son herramientas efectivas para resolver problemas de selección de proyectos con restricciones complejas.
\end{itemize}


\vspace{10cm}


\section*{Recursos y créditos}

\begin{itemize}
    \item \textbf{Código fuente:} \href{https://github.com/MateoTVara/UTP/blob/main/docs/C8/Investigacion_Operativa/S18-Evaluacion/evaluacion.pdf}{Repositorio GitHub - Investigación Operativa}
    \item \textbf{Carátula por:} \href{https://github.com/1nfinit0}{1nfinit0 en GitHub}
\end{itemize}


\end{document}