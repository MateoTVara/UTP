\documentclass[12pt]{article}

%===============================
%
%          📦 Paquetes
%
%===============================

\usepackage[a4paper, top=2cm, bottom=2cm, left=2.5cm, right=2.5cm]{geometry}
\usepackage[spanish]{babel}
\usepackage[utf8]{inputenc}
\usepackage{amsmath}
\usepackage{multicol}
\usepackage{graphicx}
\usepackage{hyperref}
\usepackage{booktabs}
\usepackage{pgfplots}
\pgfplotsset{compat=1.18}

\title{
  \vspace{2cm}
  \pagenumbering{gobble}
  \includegraphics[width=5cm]{../assets/logo-utp.png} \\
  \vspace{1cm}
  \textbf{Universidad Tecnológica del Perú} \\
  \vspace{2cm}
  \textbf{Investigación Operativa} \\
  \vspace{1cm}
  \large \textbf{S08 - Evaluación}
}
\author{
  \textbf{Torres Vara, Mateo Nicolas} - \texttt{U24308542} \\
  \texttt{Sección 36373}
}



\begin{document}
\maketitle
\begin{center}

  Docente: Alberto Andre Reyna Alcantara

\end{center}

%======================================
%
%          📚 Inicio del documento
%
%======================================

\newpage
\section*{Ejercicio 1}
\begin{table}[h]
  \centering
  \begin{tabular}{|c|c|c|c|}
  \hline
            & A   & B   & Máximo \\ \hline
  Calorías  & 100 & 150 & 2400   \\ \hline
  Proteínas & 18  & 20  & 400    \\ \hline
  Precio    & 150 & 210 &        \\ \hline
  \end{tabular}
  \caption{Variables y restricciones}
  \label{tab:Ejercicio1}
\end{table}

\begin{center}
\begin{tikzpicture}
  \begin{axis}[
    axis lines = middle,
    width=15cm, height=12cm,
    grid=both,
    xmin = 0, xmax= 27,
    ymin = 0, ymax = 22,
    xtick = {1, 3,..., 27},
    ytick = {1, 3,..., 22},
    xticklabel pos=right,
    yticklabel pos=right,
    tick label style={font=\scriptsize}
  ]

  % Recta 100x + 150y = 2400
  \addplot[red, very thick, domain=0:27] {-2/3*x + 16};
  \addlegendentry{$100x + 150y \leq 2400$}
  \addplot[red, mark=*, only marks, mark size=2pt, forget plot] coordinates {(14.28571, 7.14286) (11.42857, 9.71429)};
  % Recta 100x + 150y = 2500
  \addplot[red, thin, dashed, domain=0:27, forget plot] {-2/3*x + 50/3};
  % Recta 100x + 150y = 2600
  \addplot[red, thin, dashed,domain=0:27, forget plot] {-2/3*x + 52/3};

  % Recta 18x + 20y = 400
  \addplot[blue, very thick, domain=0:27] {-9/10*x + 20};
  \addlegendentry{$18x + 20y \leq 400$}
  \addplot[blue, mark=*, only marks, mark size=2pt, forget plot] coordinates {(19.28571, 3.14286) (21.42857, 1.71429)};
  % Recta 18x + 20y = 410
  \addplot[blue, thin, dashed, domain=0:27, forget plot] {-9/10*x + 41/2};
  % Recta 18x + 20y = 420
  \addplot[blue, thin, dashed, domain=0:27, forget plot] {-9/10*x + 21};

  \addplot[purple, mark=*, only marks, mark size=2pt] coordinates {(17.14286, 4.57143)};

  \fill[gray!30, opacity=0.7]
    (0,0) --
    (0,16) --
    (17.14286, 4.57143) --
    (22.22222, 0) --
    cycle;

  \addplot[black, mark=*, only marks, mark size=2pt] coordinates {
    (0,0) 
    (0,16) 
    (22.22222, 0)
  };

  \node at (axis cs: 0,0) [above right] {\((0,0)\)};
  \node at (axis cs: 0,16) [above right] {\((0,16)\)};
  \node at (axis cs: 17.14,4.57) [above right] {\((17.14,4.57)\)};
  \node at (axis cs: 22.22,0) [above right] {\((22.22,0)\)};

  \end{axis}
\end{tikzpicture}
\end{center}

\[
\begin{array}{c|c}
  \begin{array}{l c r}
    \text{Max} \, Z & = & 150x_1 + 210x_2 \\\\
    (0,0) & = & 3360 \\\\
    (0,16) & = & 3360 \\\\
    (17.14,4.57) & = & 3531.43 \\\\
    (22.\bar{22},0) & = & 3333.\bar{33}
  \end{array}
  &
  \begin{array}{c}
    \overset{\text{Precio Dual R1}\vphantom{\Big|}}{
    \begin{array}{c c r c r}
      100x_1 + 150x_2 & \leq & 2500 &&\\
      Z' & = & 3642.8571 & \mid & 1.14  \\\\
      100x_1 + 150x_2 & \leq & 2600 &&\\
      Z' & = & 3754.2864 & \mid & 1.14  \\\\
    \end{array}} \\
    \hline \\
    \overset{\text{Precio Dual R2}\vphantom{\Big|}}{
    \begin{array}{c c r c r}
      18x_1 + 20x_2 & \leq & 410 &&\\
      Z' & = & 3552.8571 & \mid & 2.14  \\\\
      18x_1 + 20x_2 & \leq & 420 &&\\
      Z' & = & 3754.2864 & \mid & 2.14  \\\\
    \end{array}}
  \end{array}
\end{array}
\]

\[\overset{\text{Rango de sensibilidad de recursos}\vphantom{\Big|}}{
  \begin{array}{c|c}
    \begin{array}{c c c}
      100x_1 + 150x_2 & \leq & 2400 \\[6pt]
      (0,20) & = & 3000 \\[6pt]
      (24,0) & = & 2222.\bar{2} \\[6pt]
      2222.\bar{2} & \leq \text{R1} \leq & 3000 \\[6pt]
    \end{array}
    &
    \begin{array}{c c c}
      18x_1 + 20x_2 & \leq & 400 \\[6pt]
      (0,20) & = & 432 \\[6pt]
      (22.\bar{2},0) & = & 320 \\[6pt]
      320 & \leq \text{R2} \leq & 432 \\[6pt]
    \end{array}
  \end{array}
}\]

\vspace{1cm}
\noindent\rule{\textwidth}{0.4pt}
\vspace{0.5cm}

\[\overset{\text{Rango de sensibilidad para los coeficientes de la función objetivo}\vphantom{\Big|}}{
  \begin{array}{c|c}
    \begin{array}{c c c c c}
      m & = & \dfrac{\left|C1\right|}{\left|C2\right|} & = & \dfrac{150}{210} \\\\
      \dfrac{10}{15} & < & \dfrac{\left|C1\right|}{210} & < & \dfrac{18}{20} \\\\
      140 & < & \left|C1\right| & < & 189 \\\\
    \end{array}
    &
    \begin{array}{c c c c c}
      \dfrac{10}{15} & < & \dfrac{150}{\left|C2\right|} & < & \dfrac{18}{20} \\\\
      & C2 & < & 225 & \\\\
      & 166.\bar{6} & < & C2 & \\\\
      166.\bar{6} & < & \left|C2\right| & < & 225 
    \end{array}
  \end{array}
}\]

\subsection*{Interpretaciones}
\begin{itemize}
  \item El precio dual de la restricción 1 (calorías) es 1.14, lo que significa que por cada unidad adicional de 
  calorías permitida en la dieta, el valor óptimo de la función objetivo (Z) aumentará en aproximadamente 1.14 unidades 
  monetarias, siempre y cuando las demás condiciones permanezcan constantes.
  \item El precio dual de la restricción 2 (proteínas) es 2.14, lo que indica que por cada unidad adicional de proteínas 
  permitida en la dieta, el valor óptimo de la función objetivo (Z) aumentará en aproximadamente 2.14 unidades monetarias
  , manteniendo las demás condiciones constantes.
  \item El rango de sensibilidad para los coeficientes de los recursos indica que:
    \begin{itemize}
      \item Para la restricción 1 (calorías), el coeficiente puede variar entre \(2222.\bar{2}\) y 3000 sin afectar la solución óptima actual.
      \item Para la restricción 2 (proteínas), el coeficiente puede variar entre aproximadamente 320 y 432 sin afectar la solución óptima actual.
    \end{itemize}
  \item El rango de sensibilidad para los coeficientes de la función objetivo indica que:
    \begin{itemize}
      \item El coeficiente de la variable \(x_1\) (alimento A) puede variar entre 140 y 189 sin cambiar la solución óptima actual.
      \item El coeficiente de la variable \(x_2\) (alimento B) puede variar entre \(166.\bar{6}\) y 225 sin cambiar la solución óptima actual.
    \end{itemize}
\end{itemize}





\newpage
\section*{Ejercicio 2}

\begin{table}[h]
  \centering
  \begin{tabular}{|c|c|c|c|}
  \hline
            & A   & B  & Disponibilidad \\ \hline
  Maquinas  & 5.5 & 15.5 & 750   \\ \hline
  T. a Mano & 10   & 12  & 900     \\ \hline
  Beneficio & 320 & 400 & \\ \hline
  \end{tabular}
  \caption{Variables y restricciones}
  \label{tab:Ejercicio2}
\end{table}

\begin{center}
\begin{tikzpicture}
  \begin{axis}[
    axis lines = middle,
    width=15cm, height=12cm,
    grid=both,
    xmin = 0, xmax= 180,
    ymin = 0, ymax = 100,
    xtick = {0, 20,..., 180},
    ytick = {0, 20,..., 100},
    xticklabel pos=right,
    yticklabel pos=right,
    tick label style={font=\scriptsize}
  ]

  % Recta 5.5x + 15.5y = 750
  \addplot[red, very thick, domain=0:180] {-(5.5/15.5)*x + 750/15.5};
  \addlegendentry{$5.5x + 15.5y \leq 750$}
  \addplot[red, mark=*, only marks, mark size=2pt, forget plot] coordinates {(42.13483, 39.88764) (28.65169, 51.1236)};
  % Recta 5.5x + 15.5y = 850
  \addplot[red, thin, dashed, domain=0:180, forget plot] {-(5.5/15.5)*x + 850/15.5};
  % Recta 5.5x + 15.5y = 950
  \addplot[red, thin, dashed, domain=0:180, forget plot] {-(5.5/15.5)*x + 950/15.5};

  % Recta 10x + 12y = 900
  \addplot[blue, very thick, domain=0:180] {-(10/12)*x + 75};
  \addlegendentry{$10x + 12y \leq 900$}
  \addplot[blue, mark=*, only marks, mark size=2pt, forget plot] coordinates {(73.03371, 22.47191) (90.44944, 16.29213)};
  % Recta 10x + 12y = 1000
  \addplot[blue, thin, dashed, domain=0:180, forget plot] {-(10/12)*x + 83.33333};
  % Recta 10x + 12y = 1100
  \addplot[blue, thin, dashed, domain=0:180, forget plot] {-(10/12)*x + 91.66667};

  \addplot[purple, mark=*, only marks, mark size=2pt] coordinates {(55.61798, 28.65169)};

  \fill[gray!30, opacity=0.7]
    (0,0) --
    (0,48.3871) --
    (55.61798, 28.65169) --
    (90,0) --
    cycle;

  \addplot[black, mark=*, only marks, mark size=2pt] coordinates {
    (0,0) 
    (0,48.3871) 
    (90,0)
  };

  \node at (axis cs: 0,0) [above right] {\((0;0)\)};
  \node at (axis cs: 0,48.39) [above right] {\((0;48.39)\)};
  \node at (axis cs: 55.62,28.65) [above right] {\((55.62;28.65)\)};
  \node at (axis cs: 90,0) [above right] {\((90;0)\)};

  \end{axis}
\end{tikzpicture}
\end{center}

\[
\begin{array}{c|c}
  \begin{array}{l c r}
    \text{Max} \, Z & = & 320x_1 + 400x_2 \\\\
    (0,0) & = & 0 \\\\
    (0,48.39) & = & 19354.84 \\\\
    (55.62,28.65) & = & 29258.4296 \\\\
    (90,0) & = & 28800
  \end{array}
  &
  \begin{array}{c}
    \overset{\text{Precio Dual R1}\vphantom{\Big|}}{
    \begin{array}{c c r c r}
      5.5x_1 + 15.5x_2 & \leq & 850 &&\\
      Z' & = & 29438.2016 & \mid & 1.79  \\\\
      5.5x_1 + 15.5x_2 & \leq & 950 &&\\
      Z' & = & 29617.9808 & \mid & 1.79  \\\\
    \end{array}} \\
    \hline \\
    \overset{\text{Precio Dual R2}\vphantom{\Big|}}{
    \begin{array}{c c r c r}
      10x_1 + 12x_2 & \leq & 1000 &&\\
      Z' & = & 32359.5512 & \mid & 31.01  \\\\
      10x_1 + 12x_2 & \leq & 1100 &&\\
      Z' & = & 35460.6728 & \mid & 31.01  \\\\
    \end{array}}
  \end{array}
\end{array}
\]

\[\overset{\text{Rango de sensibilidad de recursos}\vphantom{\Big|}}{
  \begin{array}{c|c}
    \begin{array}{c c c}
      5.5x_1 + 15.5x_2 & \leq & 750 \\[6pt]
      (0,75) & = & 1162.5 \\[6pt]
      (90,0) & = & 495 \\[6pt]
      495 & \leq \text{R1} \leq & 1162.5 \\[6pt]
    \end{array}
    &
    \begin{array}{c c c}
      10x_1 + 12x_2 & \leq & 900 \\[6pt]
      (136.36,0) & = & 1363.6364 \\[6pt]
      (0,48.39) & = & 580.68 \\[6pt]
      580.68 & \leq \text{R2} \leq & 1363.6364 \\[6pt]
    \end{array}
  \end{array}
}\]

\vspace{1cm}
\noindent\rule{\textwidth}{0.4pt}
\vspace{0.5cm}

\[\overset{\text{Rango de sensibilidad para los coeficientes de la función objetivo}\vphantom{\Big|}}{
  \begin{array}{c|c}
    \begin{array}{c c c c c}
      m & = & \dfrac{\left|C1\right|}{\left|C2\right|} & = & \dfrac{320}{400} \\\\
      \dfrac{5.5}{15.5} & < & \dfrac{\left|C1\right|}{400} & < & \dfrac{10}{12} \\\\
      141.93 & < & \left|C1\right| & < & 333.\bar{3} \\\\
    \end{array}
    &
    \begin{array}{c c c c c}
      \dfrac{5.5}{15.5} & < & \dfrac{320}{\left|C2\right|} & < & \dfrac{10}{12} \\\\
      & C2 & < & 901.\bar{81} & \\\\
      & 384 & < & C2 & \\\\
      384 & < & \left|C2\right| & < & 901.\bar{81} 
    \end{array}
  \end{array}
}\]

\subsection*{Interpretaciones}
\begin{itemize}
  \item El precio dual de la restricción 1 (máquinas) es 1.79, lo que significa que por cada unidad adicional de 
  tiempo de máquina permitida, el valor óptimo de la función objetivo (Z) aumentará en aproximadamente 1.79 unidades 
  monetarias, siempre y cuando las demás condiciones permanezcan constantes.
  \item El precio dual de la restricción 2 (tiempo a mano) es 31.01, lo que indica que por cada unidad adicional de tiempo
  a mano permitida, el valor óptimo de la función objetivo (Z) aumentará en aproximadamente 31.01 unidades monetarias,
  manteniendo las demás condiciones constantes.
  \item El rango de sensibilidad para los coeficientes de los recursos indica que:
    \begin{itemize}
      \item Para la restricción 1 (máquinas), el coeficiente puede variar entre 495 y 1162.5 sin afectar la solución óptima actual.
      \item Para la restricción 2 (tiempo a mano), el coeficiente puede variar entre aproximadamente 580.68 y 1363.64 sin afectar la solución óptima actual.
    \end{itemize}
  \item El rango de sensibilidad para los coeficientes de la función objetivo indica que:
    \begin{itemize}
      \item El coeficiente de la variable \(x_1\) (producto A) puede variar entre 141.93 y 333.33 sin cambiar la solución óptima actual.
      \item El coeficiente de la variable \(x_2\) (producto B) puede variar entre 384 y 901.81 sin cambiar la solución óptima actual.
    \end{itemize}
\end{itemize}

\newpage
\section*{Recursos y créditos}

\begin{itemize}
    \item \textbf{Código fuente:} \href{https://github.com/MateoTVara/UTP/tree/main/docs/C8/Investigacion_Operativa/S08-Evaluacion}{Repositorio GitHub - Investigación Operativa}
    \item \textbf{Carátula por:} \href{https://github.com/1nfinit0}{1nfinit0 en GitHub}
\end{itemize}




\end{document}