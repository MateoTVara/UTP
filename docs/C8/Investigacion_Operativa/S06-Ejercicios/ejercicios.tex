\documentclass[12pt]{article}

%===============================
%
%          📦 Paquetes
%
%===============================

\usepackage[a4paper, top=3cm, bottom=4cm, left=2.5cm, right=2.5cm]{geometry}
\usepackage[spanish]{babel}
\usepackage[utf8]{inputenc}
\usepackage{amsmath}
\usepackage{multicol}
\usepackage{graphicx}
\usepackage{hyperref}
\usepackage{booktabs}
\usepackage{pgfplots}
\pgfplotsset{compat=1.18}

\title{
  \vspace{2cm}
  \pagenumbering{gobble}
  \includegraphics[width=5cm]{../assets/logo-utp.png} \\
  \vspace{1cm}
  \textbf{Universidad Tecnológica del Perú} \\
  \vspace{2cm}
  \textbf{Investigación Operativa} \\
  \vspace{1cm}
  \large \textbf{S06 - Ejercicios}
}
\author{
  \textbf{Torres Vara, Mateo Nicolas} - \texttt{U24308542} \\
  \texttt{Sección 36373}
}



\begin{document}
\maketitle
\begin{center}

  Docente: Alberto Andre Reyna Alcantara

\end{center}

%======================================
%
%          📚 Inicio del documento
%
%======================================

\newpage
\pagenumbering{arabic}
\section*{Ejercicio 1}
\noindent Una compañía fabrica cada uno de sus productos en los siguientes 3 procesos de manera
secuencial. Los tiempos en minutos y utilidad de cada producto se presentan la siguiente tabla: \\

\begin{center}
\begin{tabular}{|l|l|l|l|l|}
  \hline
  & Proceso 1 (min) & Proceso 2 (min) & Proceso 3 (min) & Utilidad (S/.) \\
  \hline
  Proucto 1 & 10 & 6 & 8 & 2 \\
  Producto 2 & 5 & 20 & 10 & 3 \\
  \hline
  Disponibilidad & 12 horas & 10 horas & 10 horas & \\
  \hline
\end{tabular}
\end{center}

\begin{itemize}
  \item[a)] Formular el modelo y resolver utilizando el método gráfico.
  \item[b)] Si pudiera conseguir tiempo adicional en algún proceso, ¿de cuál sería?, y ¿por qué?
  \item[c)] Indique el rango de sensibilidad de los coeficientes objetivos
  \item[d)] Indique el rango de sensibilidad de los recursos disponibles
\end{itemize}

\[
\begin{array}{c|c|c}
  \begin{array}{l}
    \textbf{No Vinculante} \\
    \textbf{Vinculante} \\
    \textbf{Vinculante} \\ 
  \end{array}
  &
  \begin{array}{l c r}
    10x_1 +5x_2 & \leq & 720 \\
    6x_1 +20x_2 & \leq & 600 \\
    8x_1 +10x_2 & \leq & 600 \\
  \end{array}
  &
  \begin{array}{l}
    (0; 144) (72; 0) \\
    (0; 30) (100; 0) \\
    (0; 60) (75; 0) \\
  \end{array}
\end{array}
\]

\begin{center}
\begin{tikzpicture}
  \begin{axis}[
      axis lines = middle,
      xlabel = {$x_1$},
      ylabel = {$x_2$},
      xtick={0,10,20,30,40,50,60,70,80},
      ytick={0,10,20,30,40,50,60,70,80},
      ymin=0,
      ymax=80,
      xmin=0,
      xmax=80,
      grid=both,
      width=12cm,
      height=10cm,
      legend pos=north east,
    ]
    
    % Proceso 1
    \addplot[
      domain=0:72,
      samples=100,
      color=blue,
    ]
    {(-10*x + 720)/5};
    \addlegendentry{Proceso 1}
    
    % Proceso 2
    \addplot[
      domain=0:100,
      samples=100,
      color=red,
    ]
    {(-6*x + 600)/20};
    \addlegendentry{Proceso 2}
    
    % Proceso 3
    \addplot[
      domain=0:75,
      samples=100,
      color=green,
    ]
    {(-8*x + 600)/10};
    \addlegendentry{Proceso 3}
    
    % Sombra de la región factible
    \fill[gray!30, opacity=0.5] (0,0) -- (0,30) -- (60,12) -- (70,4) -- (72,0) -- cycle;
    
    % Puntos de intersección
    \filldraw[black] (0,30) circle (2pt) node[anchor=south west] {};
    \filldraw[black] (72,0) circle (2pt) node[anchor=north west] {};
    \filldraw[black] (70,4) circle (2pt) node[anchor=north west] {};
    \filldraw[black] (60,12) circle (2pt) node[anchor=north west] {};
    
  \end{axis}
\end{tikzpicture}
\end{center}

\newpage
\noindent Se puede conseguir más tiempo en el Proceso 1, ya que es el único recurso no vinculante.
Exactamente se puede conseguir hasta 60 minutos adicionales (1 hora).

\vspace{1cm}

\[
\begin{array}{c|c}
  \begin{array}{l c r}
    Max \ Z & = & 2x_1 + 3x_2 \\
    (0; 0) & = & 0 \\
    (0; 30) & = & 90 \\
    (60; 12) & = & 156 \\
    (72; 4) & = & 152 \\
    (72; 0) & = & 144 \\
  \end{array}
  &
  \begin{array}{c c c}
    (60; 12) & R1 & (+\infty)\\
    660 & 720 & \\
    \hline
    (70; 4) & R2 & (0; 60) \\
    500 & 600 & 1200 \\
    \hline
    (0; 30) & R3 & (67.06; 9.88)\\
    300 & 600 & 635.28
  \end{array}
\end{array}
\]

\vspace{1cm}

\[ \dfrac{6}{20} \leq \dfrac{\left|C1\right|}{\left|C2\right|} \leq \dfrac{8}{10} \]

\[
\begin{array}{c|c}
  \begin{array}{c c c c c}
    \dfrac{6}{20} & \leq & \dfrac{C1}{3} & \leq & \dfrac{8}{10} \\\\
    \dfrac{18}{20} & \leq & C1 & \leq & \dfrac{24}{10} \\\\
    0.9 & \leq & C1 & \leq & 2.4
  \end{array}
  &
  \begin{array}{c c c c c}
    \dfrac{6}{20} & \leq & \dfrac{2}{C2} & \leq & \dfrac{8}{10} \\\\
    \dfrac{20}{8} & \leq & C2 &\leq & \dfrac{40}{6} \\\\
    2.5 & \leq & C2 & \leq & 6.\overline{66} 
  \end{array}
\end{array}
\]






\newpage
\section*{Ejercicio 2}
\noindent Una empresa produce los alimentos A y B a partir de los ingredientes 1, 2 y 3. El precio de venta es de \$ $10.5$
por cada caja de 1 kilo de A y de \$ $14.5$ por cada caja de 1 kilo de B. El costo del envase es de \$ $1.5$ por 
caja. Cada uno de los alimentos contiene tres ingredientes, a continuación, se presenta la información relevante:

\begin{center}
\begin{tabular}{|l|l|l|l|}
  \hline
  Alimento & Ingrediente 1 & Ingrediente 2 & Ingrediente 3 \\
  \hline
  A & 300gr & 400gr & 300gr \\
  \hline
  B & 250gr & 500gr & 250gr \\
  \hline
  Disponibilidad & 1000kg & 1400kg & 900jg \\
  \hline
  Costo & \$ 4 / kg & \$ 3 / kg & \$ 2 / kg \\
  \hline
\end{tabular}
\end{center}

\noindent La demanda máxima es de 2000 cajas para el alimento A y de 1500 cajas para el alimento B. La capacidad de
producción total es de 3200 cajas. Determinar la cantidad a producir de cada producto para maximizar la utilidad.

\begin{itemize}
  \item [a)] Formular el modelo y resolver utilizando el método gráfico.
  \item [b)] Si pudiera conseguir una cantidad adicional de un ingrediente, ¿de cuál sería?, y ¿por qué?
  \item [c)] Indique el rango de sensibilidad de los coeficientes objetivos
  \item [d)] Indique el rango de sensibilidad de los recursos disponibles
\end{itemize}

\begin{center}
\[
\begin{array}{c|c|c}
  \begin{array}{l}
    \textbf{No Vinculante} \\
    \textbf{Vinculante} \\
    \textbf{No Vinculante} 
  \end{array}
  &
  \begin{array}{l c r}
    0.3x_1 + 0.25x_2 & \leq & 1000 \\
    0.4x_1 + 0.5x_2 & \leq & 1400 \\
    0.3x_1 + 0.25x_2 & \leq & 900 
  \end{array}
  &
  \begin{array}{l}
    (0; 4000) (3333.\overline{33}; 0) \\
    (0; 2800) (3500; 0) \\
    (0; 3600) (3000; 0) \\
  \end{array}
\end{array}
\]
\[\textbf{No Vinculante} \quad x_1 \leq 2000 \quad | \quad x_2 \leq 1500 \quad | \quad x_1 + x_2 \leq 3200 \quad \textbf{No Vinculante}\]
\hspace{-0.75cm}\textbf{Vinculante}
\end{center}

\begin{center}
\begin{tikzpicture}
  \begin{axis}[
      axis lines = middle,
      xlabel = {$x_1$},
      ylabel = {$x_2$},
      xtick={0,500,1000,1500,2000,2500,3000},
      ytick={0,500,1000,1500,2000,2500,3000,3500},
      ymin=0,
      ymax=4000,
      xmin=0,
      xmax=3500,
      grid=both,
      width=15cm,
      height=18cm,
      legend pos=north east,
    ]
    
    % Ingrediente 1
    \addplot[
      domain=0:3333.33,
      samples=100,
      color=blue,
    ]
    {(-0.3*x + 1000)/0.25};
    \addlegendentry{Ingrediente 1}
    
    % Ingrediente 2
    \addplot[
      domain=0:3500,
      samples=100,
      color=red,
    ]
    {(-0.4*x + 1400)/0.5};
    \addlegendentry{Ingrediente 2}
    
    % Ingrediente 3
    \addplot[
      domain=0:3000,
      samples=100,
      color=green,
    ]
    {(-0.3*x + 900)/0.25};
    \addlegendentry{Ingrediente 3}
    
    % Capacidad de producción
    \addplot[
      domain=0:3200,
      samples=100,
      color=orange,
    ]
    {3200 - x};
    \addlegendentry{Capacidad de producción}
    
    % Demanda máxima A (vertical line at x=2000)
    \addplot[
      color=purple,
      thick,
    ]
    coordinates {(2000,0) (2000,4000)};
    \addlegendentry{Demanda máxima A}
    
    % Demanda máxima B (horizontal line at y=1500)
    \addplot[
      color=brown,
      thick,
    ]
    coordinates {(0,1500) (4000,1500)};
    \addlegendentry{Demanda máxima B}
    
    % Sombra de la región factible
    \fill[gray!30, opacity=0.5] (0,0) -- (0,1500) -- (1625,1500) -- (2000,1200) -- (2000,0) -- cycle;
  \end{axis}
\end{tikzpicture}
\end{center}

\noindent Se podría conseguir cantidad adicional tanto del ingrediente 1 como del ingrediente 3, ya que ambos son recursos no vinculantes.
Sin embargo, se recomienda conseguir más cantidad del ingrediente 1, ya que es el que tiene menor disponibilidad (1000 kg frente a 900 kg del ingrediente 3).

\newpage
\[
\begin{array}{c|c}
  \begin{array}{l c r}
    Max \ Z & = & 6x_1 + 10x_2 \\\\
    (0; 0) & = & 0 \\\\
    (0; 1500) & = & 15000 \\\\
    (1625; 1500) & = & 24750 \\\\
    (2000; 1200) & = & 24000 \\\\
    (2000; 0) & = & 12000
  \end{array}
  &
  \begin{array}{c c c}
    (1625; 1500) & R1 & (+\infty)\\
    862.5 & 1000 & \\
    \hline
    (0; 1500) & R2 & (1750; 1500) \\
    750 & 1400 & 1450 \\
    \hline
    (1625; 1500) & R3 & (+\infty)\\
    862.5 & 900 & \\
    \hline
    (1625; 1500) & R4 & (+\infty)\\
    862.5 & 2000 & \\
    \hline
    (2000; 1200) & R5 & (0; 2000)\\
    1200 & 1500 & 2800
  \end{array}
\end{array}
\]

\vspace{1cm}

\[ \dfrac{0.4}{0.5} \leq \dfrac{\left|C1\right|}{\left|C2\right|} \leq \dfrac{0.3}{0.25} \]

\[
\begin{array}{c|c}
  \begin{array}{c c c c c}
    \dfrac{0.4}{0.5} & \leq & \dfrac{C1}{10} & \leq & \dfrac{0.3}{0.25} \\\\
    \dfrac{4}{0.5} & \leq & C1 & \leq & \dfrac{3}{0.25} \\\\
    8 & \leq & C1 & \leq & 12
  \end{array}
  &
  \begin{array}{c c c c c}
    \dfrac{0.4}{0.5} & \leq & \dfrac{6}{C2} & \leq & \dfrac{0.3}{0.25} \\\\
    \dfrac{1.5}{0.3} & \leq & C2 & \leq & \dfrac{3}{0.4} \\\\
    5 & \leq & C2 & \leq & 7.5
  \end{array}
\end{array}
\]

\end{document}